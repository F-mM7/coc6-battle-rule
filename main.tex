\documentclass{jsarticle}
\usepackage{enumitem}

\title{CoC6 戦闘ルールメモ}
\author{F}

\begin{document}
\maketitle
\section{はじめに}
\subsection*{お断り}
この記事は非公式です.
ルールブック\cite{1130282268636976512}不所持でのプレイを助長する意図はありません.
怒られたら消します.

\subsection*{概要}
ルールブック\cite{1130282268636976512}の
主に受け流し(p.66)のルールについて参照しまとめる.

\subsection*{表記について}
先述の理由から,
不要な詳細については省いて記す.
議論が必要と思われる個所については詳細に書く.

またルールブックでは技能は$\langle\cdot\rangle$を用いて書かれるが,
筆者の趣味で【$\cdot$】を用いて表記する.


\section{ラウンド毎にできること}
筆者の解釈によりまとめる.
\begin{description}[labelwidth=10em]
    \item[剣
    を持っている場合]
        受け流し and ( 【回避】 or 攻撃 )
    \item[その他]
        ( 受け流し and 【回避】 ) or 攻撃
\end{description}
\subsection*{実際の記述}
上で``剣''とした項目は正式にはフォイル,レイピア,大抵の刀剣,サーベルとある.
ちゃんと戦闘用に造られた武器という意味だろうか.

剣の項目に攻撃と受け流しが両方できる,
火器の項目に両方はできないとあり,
その他の項目には記載がない.
また,
受け流しと【回避】は両方できるとある.
また,
$\langle\mbox{回避}\rangle$(p.76)に【回避】と攻撃の両方はできないとある.
則ち素手の場合や剣や火器ではない武器を用いる場合について,
受け流しと攻撃の両方が行えるのかは明記されていない.

\section{受け流し}
戦闘に参加するキャラクターの集合を$C$とする.
キャラクター$x\in C$は各ラウンド開始時に任意のキャラクター$y\in C$を1つ宣言する.

$y$が$x$に攻撃したとき,
$x$はラウンド毎に1度,
受け流しを試みてよい.
成否を適切な技能で判定し,
成功の場合$x$は攻撃を受けず,
失敗の場合は攻撃を受ける.

ただし受け流しが物理的に不可能である場合,
受け流しは失敗する.
例えば武器による攻撃を素手で受ける場合や銃による攻撃の場合など.

物を用いて受け流しをする場合,
攻撃によるダメージだけ耐久値が減る.
超過した分は$x$自身が受ける.

\subsection*{情報の公開について}
$y(x)$が$C$に公開されるか否かについて記述がない.
一定のINTを持つ$z\in C$は$y(x)$を知ることができていいとすべきか.

\section{【マーシャルアーツ】(p.84)}
【マーシャルアーツ】は特定の攻撃を強化するが,
これと独立に,
受け流しも強化する.

この技能を持つ者は,
受け流す対象をその攻撃の直前に選んでよく,
ラウンド開始時に宣言する必要がないとある.

KP時は以下のように遂行するのがよいだろう.

$x$はラウンド毎に1度,
$y$の変更を試みてよい.
成否は【マーシャルアーツ】判定に依る.

\bibliographystyle{jplain}
\bibliography{main}

\end{document}
